\providecommand{\pdfxopts}{a-1b,cyrxmp}
\providecommand{\thisyear}{2020}
\immediate\write18{rm \jobname.xmpdata}%  uncomment for Unix-based systems
\begin{filecontents*}{\jobname.xmpdata}
\Title{Датчики. Лабораторная работа №3\textemdash\thisyear}
\Author{Артем Николаевич Прокшин}
\Creator{pdfTeX + pdfx.sty with options \pdfxopts }
\Subject{Лабораторная работа №4
\sep
        измерение напряжения сети}
\Keywords{операционный усилитель, измерение напряжения сети, ЛЭТИ}
\CoverDisplayDate{март \thisyear}
\CoverDate{2020-04-08}
\Copyrighted{True}
\Copyright{Public Domain}
\CopyrightURL{http://github.com/trot-t}
\Creator{pdfTeX + pdfx.sty with options \pdfxopts }
\end{filecontents*}

\documentclass[a4paper,12pt]{article}

\pdfcompresslevel=9

\usepackage[\pdfxopts]{pdfx}[2016/03/09]
\PassOptionsToPackage{obeyspaces}{url}
\let\tldocrussian=1  % for live4ht.cfg


\usepackage{extsizes}
\usepackage[left=15mm, top=20mm, right=15mm, bottom=20mm, nohead, footskip=7mm]{geometry} % настройки полей документа

\usepackage[T2A]{fontenc}
\usepackage[utf8]{inputenc}
\usepackage[english,russian]{babel}
\usepackage{tikz}
\usetikzlibrary{positioning}
\usepackage[european,cuteinductors,smartlabels,siunitx]{circuitikz}

%%% Межстрочный интервал
\usepackage{setspace}

%% таблицы
\usepackage{booktabs}
%multi-row
\usepackage{multirow}

%% для кода
\usepackage{color}
\usepackage{listingsutf8}

\definecolor{lightgrey}{rgb}{0.9,0.9,0.9}
\definecolor{lightblue}{rgb}{0,0,1}

\definecolor{grey}{rgb}{0.5,0.5,0.5}
\definecolor{blue}{rgb}{0,0,1}
\definecolor{violet}{rgb}{0.5,0,0.5}

\definecolor{darkred}{rgb}{0.5,0,0}
\definecolor{darkblue}{rgb}{0,0,0.5}
\definecolor{darkgreen}{rgb}{0,0.5,0}


\lstset{%
  language=C++,%
  morekeywords={constexpr,nullptr,size_t,uint32_t,assert,override,final},%
  basicstyle=\ttfamily\footnotesize,%
  sensitive=true,%
  keywordstyle=\color{blue},%
  stringstyle=\color{darkgreen},%
  commentstyle=\color{violet},%
  showstringspaces=false,%
  tabsize=2,%
  frame=leftline,
  rulecolor=\color{lightblue},
  xleftmargin=20pt,
}

\lstset{
%extendedchars=\true,
%inputencoding=utf8x,   
  numberstyle=\tiny,
  numbers=left,
  numbersep=10pt,
  xleftmargin=20pt,
  %framesep=4.5mm,
  %framexleftmargin=2.5mm,
  framexleftmargin=5pt,
  framesep=15pt,
  fillcolor=\color{lightgrey},
}


\title{Прецизионные схемы и малошумящая аппаратура}
\author{}
% Конец преамбулы
\begin{document}
%\maketitle
%\begin{tikzpicture}
%\newcommand{\xb}{-3}
%\newcommand{\xa}{3}
%\draw[thin, ->] (-6,0) -- (6,0) node[right] {$X$};
%\draw[thin, ->] (0,-6) -- (0,6) node[left] {$Y$};
%\foreach \x\xtext in {-5/-5,5/5,{\xb}/\xb,{\xa}/{\displaystyle \frac{-b+\sqrt{b^2-4ac}}{2a}}} % 
%   \draw (\x,0.1) -- (\x,-0.1) node[below] {$\xtext$};

 %\draw[domain=-5:5, help lines, smooth]
 %       plot ({\x},{0.2*(\x-\xa)*(\x-\xb)});
%\end{tikzpicture}
%\section{Прецизионные схемы и малошумящая аппаратура}
\section{Измерение напряжения сети}

{\it Цель:} научиться измерять напряжение сети.
 
\subsection{Краткие теоретические сведения}

Для получения информации о форме и уровне напряжения фаз сети, к которой подключен фильтр используются трансформаторы ТП-321-461Р, 
обеспечивающие согласование уровней и гальваническую развязку сигналов. 
Преобразованный сигнал с трансформатора требуется снизить до уровня амплитуды не более 1,5 В и обеспечить смещение +1,5 В 
для корректной обработки при помощи АЦП микроконтроллера, т.к. АЦП может обрабатывать сигналы в диапазоне 0..3,3 В. 
Для решения этих задач применены операционные усилители OP297, имеющие диапазон питающего напряжения от $\pm2$ до $\pm20$ В, 
низкий уровень смещения не более 50 мкВ и проявившие стабильность свойств и надежность при использовании в аналогичных цепях. 

Описание работы на примере одного канала, два других работают идентично (см. Рисунок 11)

\begin{figure}[!ht]
\centering
\begin{circuitikz}[scale=0.9]
\draw
	(0,0) node[transformer core] (T) {}
	(T.A1) node[anchor=east, above] {A}
	(T.A2) node[anchor=east, above] {N}
	(T.B1) node[anchor=west] {}
	(T.B2) node[anchor=west] {}
	(T.base) node{$T_x$}
	(T.B2) --++ (1,0) to[R, l=$R_{xx}$, *-*] ++ (0,2) node(N1) {} to[R, l=$R_{xx}$, -*] ++ (2,0) node(N2) {} 
	to[D-,l=\small{$D_x$},-*] ++ (0,-2) node(N3) {} to[short] ++ (-2,0) 
	(N3) to[short] ++ (1.5,0) node(N4) {}  to[D-,l_=\small{$D_x$},*-*] ++ (0,2) node(N5) {} to[short] ++ (-1.5,0)
	($(N4) ! 0.5 ! (N5)$) ++ (2.5,0) node[op amp] (opamp) {}
	(opamp.+) |- (N4.center) 
	(opamp.-) |- (N5.center)
	(opamp.up) --++(0,0.5) node[vcc] {\small{5\,\textnormal{V}}}
	(opamp.down) --++(0,-0.5) node[vee]{\small{-5\,\textnormal{V}}}
	(opamp.out) to[short] ++ (0,3) node(N6) {} to[R,l_=$R_x$, *-*] (opamp.- |- N6) node(N7) {} to[short, -*] (opamp.- |- N5)
	(N6.center) --++ (0,1.5) node(N8) {}  to[C,l_=$C_x$] (opamp.- |- N8)  to[short] (opamp.- |- N6)
	($(N4) ! 0.5 ! (N5)$) ++ (7,0) node[op amp] (opamp1) {}
	(opamp.out |- N5) to[R,l=$R_x$,*-] ++ (1.8,0) -| (opamp1.-)
	(opamp1.out) to[R,l=$R_x$] ++ (1.5,0) node(N9) {} to[C,l=$C_x$,*-] ++ (0,-2) 
	(opamp1.+) to[R,l=$R_x$] ++ (-1.2,0) to[short] node[midway,below] {offset 1} ++ (-1,0) 
	(N9) to[short] ++ (1,0) node[above] {$U_a$} 
	;
\end{circuitikz}
\end{figure}

Напряжение с трансформатора $T_x$ поступает на резисторный делитель напряжения $R_x-R_x$, понижающий уровень на $\ldots$, 
затем на усилитель $DA_x$ и диодами $D_x$, $D_x$, ограничивающими напряжение на дифференциальном входе усилителя 
при отсутствии питания схемы. Усилитель выполнен по схеме инвертирующего усилителя с коэффициентом усиления 1 
(коэффициент задан резисторами $R_x$ и $R_x$). Параллельно резистору $R_x$ установлен конденсатор$ C_x$ для фильтрации 
высокочастотных помех.
С выхода усилителя сигнал поступает на сумматор сигналов, выполненный на операционном усилителе $DA_x$ 
и резисторах $R_x$, $R_x$ и $R_x$. Выходное напряжение сумматора является суммой (с "весом") напряжений $V_{DA}$ и $V_{offset 1}$. 
Выходное напряжение сумматора () определяется формулой (30):

$$
V_{DA} = \ldots
$$


\begin{tabular}{lll}
	где $V_{DA}$ &--& это напряжение на выводу усилителя\\
	$V_{offset 1}$ &--& напряжение в сети offset\\
	$V_{DA}$ & -- & напряжение на выводе усилителя $DA_x$
\end{tabular}

Сигнал для смещения (см. \ldots) формируется источником {$DA_x$}, который имеет выходное напряжение 3 В, 
стабилизируется и фильтруется конденсаторами $С_x$ и $С_x$, поступает на резисторный делитель $R_x-R_x$ с коэффициентом $\ldots$ , 
затем, полученные 1,5 В поступают на повторители $DA_x$ и $DA_x$, с которых берется сигнал для смещения.


Сумматор обеспечивает сложение переменного сигнала с усилителя $DA_x$ с сигналом смещения +1,5 В. 
Затем выходной сигнал сумматора фильтруется R-C фильтром низких частот $R_x-C_x$ c частотой среза 5 кГц 
(см. формулу (\ldots)) и поступает на вход АЦП микроконтроллера.
Частота среза НЧ-фильтра:

$$
f_c = \ldots
$$

\end{document}
