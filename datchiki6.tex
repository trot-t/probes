\providecommand{\pdfxopts}{a-1b,cyrxmp}
\providecommand{\thisyear}{2020}
\immediate\write18{rm \jobname.xmpdata}%  uncomment for Unix-based systems
\begin{filecontents*}{\jobname.xmpdata}
\Title{Датчики. Практическая работа №6\textemdash\thisyear}
\Author{Артем Николаевич Прокшин}
\Creator{pdfTeX + pdfx.sty with options \pdfxopts }
\Subject{Практическая работа №6
\sep
        датчики для системы управления ТАД}
\Keywords{операционный усилитель, датчики для системы управления ТАД, ЛЭТИ}
\CoverDisplayDate{март \thisyear}
\CoverDate{2020-04-08}
\Copyrighted{True}
\Copyright{Public Domain}
\CopyrightURL{https://www.researchgate.net/publication/371893460_Modification_of_the_Space_Vector_Pulse-Width_Modulation_Algorithm_for_a_Three-Phase_Inverter_Using_an_Oblique_Coordinate_System}

\end{filecontents*}

\documentclass[a4paper,12pt]{article}

\pdfcompresslevel=9

\usepackage[\pdfxopts]{pdfx}[2016/03/09]
\PassOptionsToPackage{obeyspaces}{url}
\let\tldocrussian=1  % for live4ht.cfg


\usepackage{extsizes}
\usepackage[left=15mm, top=20mm, right=15mm, bottom=20mm, nohead, footskip=7mm]{geometry} % настройки полей документа

\usepackage[T2A]{fontenc}
\usepackage[utf8]{inputenc}
\usepackage[english,russian]{babel}
\usepackage{tikz}
\usetikzlibrary{positioning}
\usepackage[european,cuteinductors,smartlabels,siunitx]{circuitikz}

%%% Межстрочный интервал
\usepackage{setspace}

%% таблицы
\usepackage{booktabs}
%multi-row
\usepackage{multirow}

%% для кода
\usepackage{color}
\usepackage{listingsutf8}

\definecolor{lightgrey}{rgb}{0.9,0.9,0.9}
\definecolor{lightblue}{rgb}{0,0,1}

\definecolor{grey}{rgb}{0.5,0.5,0.5}
\definecolor{blue}{rgb}{0,0,1}
\definecolor{violet}{rgb}{0.5,0,0.5}

\definecolor{darkred}{rgb}{0.5,0,0}
\definecolor{darkblue}{rgb}{0,0,0.5}
\definecolor{darkgreen}{rgb}{0,0.5,0}


\lstset{%
  language=C++,%
  morekeywords={constexpr,nullptr,size_t,uint32_t,assert,override,final},%
  basicstyle=\ttfamily\footnotesize,%
  sensitive=true,%
  keywordstyle=\color{blue},%
  stringstyle=\color{darkgreen},%
  commentstyle=\color{violet},%
  showstringspaces=false,%
  tabsize=2,%
  frame=leftline,
  rulecolor=\color{lightblue},
  xleftmargin=20pt,
}

\lstset{
%extendedchars=\true,
%inputencoding=utf8x,   
  numberstyle=\tiny,
  numbers=left,
  numbersep=10pt,
  xleftmargin=20pt,
  %framesep=4.5mm,
  %framexleftmargin=2.5mm,
  framexleftmargin=5pt,
  framesep=15pt,
  fillcolor=\color{lightgrey},
}


\title{Прецизионные схемы и малошумящая аппаратура}
\author{}
% Конец преамбулы
\begin{document}
%\maketitle
%\begin{tikzpicture}
%\newcommand{\xb}{-3}
%\newcommand{\xa}{3}
%\draw[thin, ->] (-6,0) -- (6,0) node[right] {$X$};
%\draw[thin, ->] (0,-6) -- (0,6) node[left] {$Y$};
%\foreach \x\xtext in {-5/-5,5/5,{\xb}/\xb,{\xa}/{\displaystyle \frac{-b+\sqrt{b^2-4ac}}{2a}}} % 
%   \draw (\x,0.1) -- (\x,-0.1) node[below] {$\xtext$};

 %\draw[domain=-5:5, help lines, smooth]
 %       plot ({\x},{0.2*(\x-\xa)*(\x-\xb)});
%\end{tikzpicture}
%\section{Прецизионные схемы и малошумящая аппаратура}
\section{Задание 6: Датчики для системы управления ТАД}

{\it Цель: Рассчитать и смоделировать датчики для системы управления ТАД} 

%\begin{table}[!ht]
%\begin{tabular}{cc} 
%\begin{minipage}[h]{0.5\linewidth} 
\begin{figure}[ht!]
\centering
\begin{circuitikz}[scale=1]
\ctikzset{bipoles/length=1.0cm}

\draw(1.25,2.65)node[nigbt,bodydiode](npn1){};% 1 ряд 
%\node[nigbt,bodydiode] (npn1) at (1.25,3.1) {};% 1 ряд 
\draw (1.25,.55) node[nigbt,bodydiode](npn4){};%1ряд
\draw (npn1.S) -- (npn4.D);

% найдем положение плюсовой шины
\path let \p1 = (npn1.D) in node(plus)  at (0,\y1) {};
%	\draw (0,0) -- (-2,0)node(A){}  to[C] (A |- plus);

	\draw (plus) to[smeter, t=V, v=$v$,american] (0,0);

\draw(2.75,2.65)node[nigbt,bodydiode](npn3){};% 2 ряд 
\draw (2.75,.55) node[nigbt,bodydiode](npn6){};% 2ряд
\draw (npn3.S) -- (npn6.D);

\draw (4.25,2.65)node[nigbt,bodydiode](npn5){};;%последний ряд 
\draw (4.25,.55) node[nigbt,bodydiode](npn2){};%последний ряд
\draw (npn5.S) -- (npn2.D);

\draw (plus.center) --(npn1.D) node[above]{1} -- (npn3.D) node[above]{3} -- (npn5.D) node[above]{5}; % плюсовая шина
\draw (0,0) -- (npn4.S) node[below]{4} -- (npn6.S) node[below]{6} -- (npn2.S) node[below]{2}; % минусовая шина

	\draw ($(npn5.S)!0.95!(npn2.D)$)   node[left]{\scriptsize$C$} to[short,*-] ++ (0.25,0) to[iloop, mirror, name=IC] ++ (1,0) --++ (2,0) to[L,american inductor] ++ (1,0) node(C) {};    %катуха С 
	\draw (IC.i) to [smeter, t=A, i=$i_C$] ++ (0,-2);
	\draw ($(npn3.S)!0.5!(npn6.D)$) node[left]{\scriptsize$B$} to[short,*-] ($(npn5.S)!0.5!(npn2.D)$) -- ++ (0.25,0)  to[short] ++ (0.75,0) to[iloop, mirror, name=IB] ++ (1.75,0) --++ (0.5,0)  
	to[L,american inductor, -*]  (C.center |- IB);  % катуха В 
	\draw (IB.i) -- (IC.i -| IB.i) to [smeter, t=A, i=$i_B$] ++ (0,-2);
	\draw ($(npn1.S)!0.05!(npn4.D)$) node[left]{\scriptsize$A$} to[short,*-] ($(npn5.S)!0.05!(npn2.D)$) -- ++ (0.25,0)  to[short] ++ (2.05,0) to[iloop, mirror, name=IA] ++ (1,0) to[L,american inductor] (C.center |- IA) node(A) {};
       \draw (IA.i) -- (IC.i -| IA.i) to [smeter, t=A, i=$i_A$] ++ (0,-2);

\draw (A.center)--(C.center);
\end{circuitikz}
	\caption{схема автономного инвертора напряжения для трех фаз ТАД и места установки датчиков}
	\label{ain}
\end{figure}
%\end{minipage}
%	}
%\end{tabular}
%\end{table}

 
\subsection{Задание на работу №6}

Рассчитать и смоделировать в программе ngspice или tina схему измерения температуры, постоянного напряжения и силы тока фаз инвертора в силовой цепи тягового aсинхронного двигателя трамвая. 
Сделать акцент на отечественную элементную базу.
Использовать трансформатор тока фирмы ЭЛТИ.
Использовать дифференциальный усилитель AD8276 или ОУ, указанными \href{http://motorcontrol.ru/wp-content/uploads/2019/08/MCB2.pdf}{здесь}.

Сигналы измерения подготовить для измерения аналогово-цифровыми преобразователями микроконтроллера К1921ВК01Т.




\renewcommand{\bibname}{}
\begin{thebibliography}{1}
%	\bibitem{lm211} \href{http://motorcontrol.ru/production/controlcards/motorcontrolboard/}{Отладочная плата MotorControlBoard К1921ВК01}
        \bibitem{ngspice} \href{https://ngspice.sourceforge.io/}{Сайт с программой ngspice}
        \bibitem{tina} \href{https://www.ti.com/tool/TINA-TI}{Сайт с программой tina (по состоянию 4 октября 2023 сайт с территории России недоступен)}
        \bibitem{k1921} \href{http://motorcontrol.ru/wp-content/uploads/2019/08/MCB2.pdf}{схема электрическая принципиальная отладочнай платы MotorControlBoard К1921ВК01}
        \bibitem{lm211} \href{https://www.ti.com/product/LM211}{LM211 Differential Comparator With Strobes, PSpice Model}

\end{thebibliography}


%http://motorcontrol.ru/production/controlcards/motorcontrolboard/
% https://habr.com/ru/company/npf_vektor/blog/389129/




\end{document}
