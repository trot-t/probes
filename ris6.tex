\documentclass[utf8x, times, 14pt]{extarticle}
%\documentclass[utf8x, times, 14pt]{extstandalone}
%\documentclass[utf8x, times, 14pt]{standalone}
\usepackage[fontsize=12]{fontsize}
%\usepackage[active,tightpage]{preview}

\usepackage[T2A]{fontenc}
\usepackage[utf8]{inputenc}
\usepackage[english,russian]{babel}
\usepackage{tikz}
\usetikzlibrary{positioning}
\usepackage[european,cuteinductors,smartlabels,siunitx]{circuitikz}

\usepackage[a4paper, margin=0.0cm]{geometry}

\begin{document}

\pdfpagewidth 3.0in
\pdfpageheight 3.1in

\hspace{-6mm}
\begin{tikzpicture}
        \draw
        (0,0) node[op amp] (opamp) {} % по координатам (0,0) операционный усилитель с меткой (opamp) и пустой {} подписью
        (opamp.-) to[R,l_=$R_1$,-o] ++ (-3,0) node(IN_P) {}  % инв.вход ОУ соединяем со входом "-o" схемы резистором R1
                                     % маркируем вход схемы узлом(node) с меткой (IN_P), маркировкой пользуемся ниже
                                     % в фигурных скобках стоит подпись(пустая) к метке
        % у метки ОУ есть якоря маркирующие выводы, (opamp.-) , (opamp.+) и (opamp.-)
        % есть якоря с азимутальной (opamp.west) и геометрической (opamp.сеnter) привязкой
        (opamp.out) to[short,-o] ++ (1,0) node(OUT_P) {}       % вывод ОУ вместе с кружком "-o"
                                          % маркируем выход схемы узлом (OUT_P)
        (opamp.-) to[short,*-] ++ (0,1.2) node (A) {} to[R,l=$R_{\textcyrillic{ОС}}$] % R обратной связи от инв. входа
                                          % не знаем на какой высоте остановились, поэтому маркируем узел буквой A
        (opamp.out |- A) to[short,-*] (opamp.out) % от (.) пересечения x-коорд. выхода ОУ и y-коорд. узла A
%       (opamp.+) --++ (-1,0) to[R,l_={$R_2$}] ++ (0,-3) node[tground] (G) {} % неинв. вход ОУ соединим с землей резистором R2
        (opamp.+) --++ (-1,0) to[short] ++ (0,-3) node[tground] (G) {} % неинв. вход ОУ соединим с землей шунтом
                                         % не знаем на какой y-коорд. остановились, поэтому маркируем узел буквой G
        (IN_P |- G) node[tground] {} to[short,-o] ++ (0,2.5) node(IN_M) {}
        %  от (.) пересечения x-коорд. входа ОУ и y-коорд. маркировки G, вход схемы маркируем (IN_M)
        ($(IN_P) !0.5! (IN_M)$) node {$U_\textcyrillic{вх}$} % посередине между узлами (+) и (-) входов -- подпись к метке
        (OUT_P |- G) node[tground] {} to[short,-o] ++ (0,2) node(OUT_M) {} % от земли до выхода схемы
        ($(OUT_P) !0.5! (OUT_M)$) node {$U_\textcyrillic{вых}$}  % справа от точки посередине между выходами схемы

        %дополняем схему инвертирующего подключения ОУ тем что нужно для активного фильтра
        (A.center) --++ (0,1.3) node(B) {} to[C,l= $C$] (opamp.out |- B) -- (opamp.out |- A)
        ;  % всегда в конце команды(в данном случае \draw) ставим точку с запятой
        % !!! это самая распространенная ошибка почему может не работать
\end{tikzpicture}

\end{document}

