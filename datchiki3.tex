\documentclass[a4paper,12pt]{article}
\usepackage{extsizes}

\usepackage[T2A]{fontenc}
\usepackage[utf8]{inputenc}
\usepackage[english,russian]{babel}
\usepackage{tikz}
\usetikzlibrary{positioning}
\usepackage[european,cuteinductors,smartlabels,siunitx]{circuitikz}

%%% Межстрочный интервал
\usepackage{setspace}

%% таблицы
\usepackage{booktabs}
%multi-row
\usepackage{multirow}

%% для кода
\usepackage{color}
\usepackage{listingsutf8}

\definecolor{lightgrey}{rgb}{0.9,0.9,0.9}
\definecolor{lightblue}{rgb}{0,0,1}

\definecolor{grey}{rgb}{0.5,0.5,0.5}
\definecolor{blue}{rgb}{0,0,1}
\definecolor{violet}{rgb}{0.5,0,0.5}

\definecolor{darkred}{rgb}{0.5,0,0}
\definecolor{darkblue}{rgb}{0,0,0.5}
\definecolor{darkgreen}{rgb}{0,0.5,0}


\lstset{%
  language=C++,%
  morekeywords={constexpr,nullptr,size_t,uint32_t,assert,override,final},%
  basicstyle=\ttfamily\footnotesize,%
  sensitive=true,%
  keywordstyle=\color{blue},%
  stringstyle=\color{darkgreen},%
  commentstyle=\color{violet},%
  showstringspaces=false,%
  tabsize=2,%
  frame=leftline,
  rulecolor=\color{lightblue},
  xleftmargin=20pt,
}

\lstset{
%extendedchars=\true,
%inputencoding=utf8x,   
  numberstyle=\tiny,
  numbers=left,
  numbersep=10pt,
  xleftmargin=20pt,
  %framesep=4.5mm,
  %framexleftmargin=2.5mm,
  framexleftmargin=5pt,
  framesep=15pt,
  fillcolor=\color{lightgrey},
}


\title{Прецизионные схемы и малошумящая аппаратура}
\author{}
% Конец преамбулы
\begin{document}
%\maketitle
%\begin{tikzpicture}
%\newcommand{\xb}{-3}
%\newcommand{\xa}{3}
%\draw[thin, ->] (-6,0) -- (6,0) node[right] {$X$};
%\draw[thin, ->] (0,-6) -- (0,6) node[left] {$Y$};
%\foreach \x\xtext in {-5/-5,5/5,{\xb}/\xb,{\xa}/{\displaystyle \frac{-b+\sqrt{b^2-4ac}}{2a}}} % 
%   \draw (\x,0.1) -- (\x,-0.1) node[below] {$\xtext$};

 %\draw[domain=-5:5, help lines, smooth]
 %       plot ({\x},{0.2*(\x-\xa)*(\x-\xb)});
%\end{tikzpicture}
%\section{Прецизионные схемы и малошумящая аппаратура}
\section{Изучение работы избирательных усилителей (активных фильтров)}

{\it Цель:} изучить работу активного фильтра, научиться измерять его статические характеристики, определять их аналитически.
 
\subsection{Краткие теоретические сведения}

Избирательными усилителями (активными фильтрами) называют усилители, которые по совокупности принимаемых сигналов
выбирают и усиливают только синусоидальные сигналы, занимающие определенный участок спектра частот.
Активные фильтры часто реализуют, используя пассивные $RC$-цепи и ОУ в качестве активного элемента.

Избирательные (селективные) свойства таких устройств (то есть их способность выделять полезный сигнал и ослаблять помехи)
характеризуется АЧХ. Избирательные усилители обладают особой формой АЧХ.

Полосу частот, в которой осуществляется усиление сигнала, называют полосой пропускания (прозрачности). Полосу частот,
в которой сигналы подавляются, называют полосой заграждения. В зависимости от взаимного расположения полос
пропускания и загряждения различают виды фильтров: нижних частот, верхних частот, полосовые пропускания,
полосовые заграждения. Значения коэффициента передачи (усиления) в полосах пропускания и заграждения 
могут значительно различаться. Поэтому обычно АЧХ фильтра представляет собой зависимость его нормированного 
коэффициента усиления $K/K_{0}$ от частоты $f$ в логарифмическом масштабе. Фильтр нижних частот без изменения
передает сигналы низкой частоты и обеспечивает затухание высокочастотных сигналов. Вид АЧХ активного фильтра 
нижних частот определяется типом $RC$-фильтра.

 На рис.\ref{active_filter} приведена схема активного фильтра, построенного на основе инвертирующего ОУ
 и интегратора. Такой фильтр представляет собой инвертирующий усилитель с постоянным коэффициентом усиления
 в полосе частот от $f_\textcyrillic{м} = 0$ до $f_\textcyrillic{с}$. Частота среза $f_\textcyrillic{с}$ с 
 которой начинается уменьшение коэффициента усиления, регулируется цепью обратной связи 
 ${\displaystyle f_\textcyrillic{с} = \frac{1}{2\pi R_{OC} C}}$. На частоте выше коэффициент усиления уменьшается
 на 20 дБ/дек, что соответствует уменьшению коэффициента усиления в два раза при удвоении частоты.
 Для получения АЧХ с более крутой характеристикой применяют каскадное соединение простых фильтров.


\begin{figure}[ht!] % [ht!] -- попытка разместить рисунок сразу в этом же месте текста.
\centering
\begin{tikzpicture}
	\draw
	(0,0) node[op amp] (opamp) {} % по координатам (0,0) операционный усилитель с меткой (opamp) и пустой {} подписью 
	(opamp.-) to[R,l_=$R_1$,-o] ++ (-3,0) node(IN_P) {}  % инв.вход ОУ соединяем со входом "-o" схемы резистором R1
	                             % маркируем вход схемы узлом(node) с меткой (IN_P), маркировкой пользуемся ниже
	                             % в фигурных скобках стоит подпись(пустая) к метке
	% у метки ОУ есть якоря маркирующие выводы, (opamp.-) , (opamp.+) и (opamp.-)
	% есть якоря с азимутальной (opamp.west) и геометрической (opamp.сеnter) привязкой
	(opamp.out) to[short,-o] ++ (1,0) node(OUT_P) {}       % вывод ОУ вместе с кружком "-o"
	                                  % маркируем выход схемы узлом (OUT_P)
	(opamp.-) to[short,*-] ++ (0,1.2) node (A) {} to[R,l=$R_{OC}$] % R обратной связи от инв. входа 
	                                  % не знаем на какой высоте остановились, поэтому маркируем узел буквой A 
	(opamp.out |- A) to[short,-*] (opamp.out) % от (.) пересечения x-коорд. выхода ОУ и y-коорд. узла A 
%	(opamp.+) --++ (-1,0) to[R,l_={$R_2$}] ++ (0,-3) node[tground] (G) {} % неинв. вход ОУ соединим с землей резистором R2
	(opamp.+) --++ (-1,0) to[short] ++ (0,-3) node[tground] (G) {} % неинв. вход ОУ соединим с землей шунтом
	                                 % не знаем на какой y-коорд. остановились, поэтому маркируем узел буквой G
	(IN_P |- G) node[tground] {} to[short,-o] ++ (0,2.5) node(IN_M) {}
	%  от (.) пересечения x-коорд. входа ОУ и y-коорд. маркировки G, вход схемы маркируем (IN_M)
	($(IN_P) !0.5! (IN_M)$) node {$U_\textcyrillic{вх}$} % посередине между узлами (+) и (-) входов -- подпись к метке
	(OUT_P |- G) node[tground] {} to[short,-o] ++ (0,2) node(OUT_M) {} % от земли до выхода схемы 
	($(OUT_P) !0.5! (OUT_M)$) node[right] {$U_\textcyrillic{вых}$}  % справа от точки посередине между выходами схемы
	
	%дополняем схему инвертирующего подключения ОУ тем что нужно для активного фильтра
	(A.center) --++ (0,1.3) node(B) {} to[C,l=C] (opamp.out |- B) -- (opamp.out |- A)
	;  % всегда в конце команды(в данном случае \draw) ставим точку с запятой
	% !!! это самая распространенная ошибка почему может не работать
\end{tikzpicture}
	\caption{схема активного фильтра низких частот при инвертирующем включении} 
	\label{active_filter} % ссылка, на которую сошлемся в тексте
\end{figure}


\begin{figure}[ht!] % [ht!] -- попытка разместить рисунок сразу в этом же месте текста.
\centering
\begin{tikzpicture}
        \draw
        (0,0) node[op amp, yscale=-1] (opamp) {} 
	(opamp.-) --++ (-1,0) --++ (0,-1.2) node(A) {} to[R,l_={$R_1$}] ++ (0,-3) node[tground] (G) {} % инв. вход ОУ соединим с землей резистором R1	
	(opamp.out) to[short,*-] (opamp.out |- A) --++ (-0.7, 0) node(B) {}
	(A.center) to[short,*-] ++ (0.7,0) node(C){} to[R,l_=$R_{OC}$] (B)
	(C.center) to[short,*-] ++ (0,-1.3) node(D) {}  to[C,l_=$C$] (B |- D) to[short,-*] (B)
	(opamp.out) to[short,-o] ++ (1,0) node(OUT_P) {}       % вывод ОУ вместе с кружком "-o"
        (OUT_P |- G) node[tground] {} to[short,-o] ++ (0,3) node(OUT_M) {} % от земли до выхода схемы 
        ($(OUT_P) !0.5! (OUT_M)$) node[right] {$U_\textcyrillic{вых}$}  % справа от точки посередине между выходами схемы
        (opamp.+) to[short,-o] ++ (-3,0) node(IN_P) {}  % инв.вход ОУ соединяем со входом "-o" схемы резистором R1
        (IN_P |- G) node[tground] {} to[short,-o] ++ (0,3.5) node(IN_M) {}
	($(IN_P) !0.5! (IN_M)$) node {$U_\textcyrillic{вх}$} % посередине между узлами (+) и (-) входов -- подпись к метке
        ;  % всегда в конце команды(в данном случае \draw) ставим точку с запятой
        % !!! это самая распространенная ошибка почему может не работать
\end{tikzpicture}
        \caption{схема активного фильтра низких частот при неинвертирующем включении}
        \label{active_filter2} % ссылка, на которую сошлемся в тексте
\end{figure}


\begin{figure}[ht!] % [ht!] -- попытка разместить рисунок сразу в этом же месте текста.
\centering
\begin{tikzpicture}
        \draw
        (0,0) node[op amp] (opamp) {} % по координатам (0,0) операционный усилитель с меткой (opamp) и пустой {} подписью
	(opamp.-) to[R,l_=$R_1$] ++ (-2.5,0) to[C,l_=$C$,-o] ++ (-2.5,0) node(IN_P) {}  % инв.вход ОУ соединяем со входом "-o" схемы резистором R1
                                     % маркируем вход схемы узлом(node) с меткой (IN_P), маркировкой пользуемся ниже
                                     % в фигурных скобках стоит подпись(пустая) к метке
        % у метки ОУ есть якоря маркирующие выводы, (opamp.-) , (opamp.+) и (opamp.-)
        % есть якоря с азимутальной (opamp.west) и геометрической (opamp.сеnter) привязкой
        (opamp.out) to[short,-o] ++ (1,0) node(OUT_P) {}       % вывод ОУ вместе с кружком "-o"
                                          % маркируем выход схемы узлом (OUT_P)
        (opamp.-) to[short,*-] ++ (0,1.2) node (A) {} to[R,l=$R_{OC}$] % R обратной связи от инв. входа
                                          % не знаем на какой высоте остановились, поэтому маркируем узел буквой A
        (opamp.out |- A) to[short,-*] (opamp.out) % от (.) пересечения x-коорд. выхода ОУ и y-коорд. узла A
%       (opamp.+) --++ (-1,0) to[R,l_={$R_2$}] ++ (0,-3) node[tground] (G) {} % неинв. вход ОУ соединим с землей резистором R2
        (opamp.+) --++ (-1,0) to[short] ++ (0,-3) node[tground] (G) {} % неинв. вход ОУ соединим с землей шунтом
                                         % не знаем на какой y-коорд. остановились, поэтому маркируем узел буквой G
        (IN_P |- G) node[tground] {} to[short,-o] ++ (0,2.5) node(IN_M) {}
        %  от (.) пересечения x-коорд. входа ОУ и y-коорд. маркировки G, вход схемы маркируем (IN_M)
        ($(IN_P) !0.5! (IN_M)$) node {$U_\textcyrillic{вх}$} % посередине между узлами (+) и (-) входов -- подпись к метке
        (OUT_P |- G) node[tground] {} to[short,-o] ++ (0,2) node(OUT_M) {} % от земли до выхода схемы
        ($(OUT_P) !0.5! (OUT_M)$) node[right] {$U_\textcyrillic{вых}$}  % справа от точки посередине между выходами схемы

        %дополняем схему инвертирующего подключения ОУ тем что нужно для активного фильтра
%        (A.center) --++ (0,1.3) node(B) {} to[C,l=C] (opamp.out |- B) -- (opamp.out |- A)
        ;  % всегда в конце команды(в данном случае \draw) ставим точку с запятой
        % !!! это самая распространенная ошибка почему может не работать
\end{tikzpicture}
	\caption{схема активного фильтра высоких частот при инвертирующем включении}
        \label{active_filter3} % ссылка, на которую сошлемся в тексте
\end{figure}


\begin{figure}[ht!] % [ht!] -- попытка разместить рисунок сразу в этом же месте текста.
\centering
\begin{tikzpicture}
        \draw
        (0,0) node[op amp, yscale=-1] (opamp) {}
	(opamp.-) --++ (-1,0) --++ (0,-1.2) node(A) {} to[R,l_={$R_1$}] ++ (0,-2) --++ (0,0.4) to[C,l={C}] ++ (0,-2)  node[tground] (G) {} % инв. вход ОУ соединим с землей резистором R1   
	(A.center) to[R,l_=$R_{OC}$,*-] (opamp.out |- A) to[short,-*] (opamp.out) 
        (opamp.out) to[short,-o] ++ (1,0) node(OUT_P) {}       % вывод ОУ вместе с кружком "-o"
        (OUT_P |- G) node[tground] {} to[short,-o] ++ (0,3) node(OUT_M) {} % от земли до выхода схемы 
        ($(OUT_P) !0.5! (OUT_M)$) node[right] {$U_\textcyrillic{вых}$}  % справа от точки посередине между выходами схемы
        (opamp.+) to[short,-o] ++ (-3,0) node(IN_P) {}  % инв.вход ОУ соединяем со входом "-o" схемы резистором R1
        (IN_P |- G) node[tground] {} to[short,-o] ++ (0,3.5) node(IN_M) {}
        ($(IN_P) !0.5! (IN_M)$) node {$U_\textcyrillic{вх}$} % посередине между узлами (+) и (-) входов -- подпись к метке
        ;  % всегда в конце команды(в данном случае \draw) ставим точку с запятой
        % !!! это самая распространенная ошибка почему может не работать
\end{tikzpicture}
        \caption{схема активного фильтра высоких частот при неинвертирующем включении}
        \label{active_filter4} % ссылка, на которую сошлемся в тексте
\end{figure}


\subsection{Последовательность выполнения задания}
\begin{itemize}
\item Собрать схему активного фильтра согласно рис. \ref{active_filter}. Напряжение на входе и выходе фильтра контролировать 
	при помощи осциллографа. Значение резистора $R_{OC}$ и ёмкости $C$ установить в соответствии с указанием
		преподавателя.
\item Снять и построить амплитудно-частотную характеристику фильтра при заданных параметрах элемента фильтра.
	Результаты измерений занести в таблицу \ref{a4x}
\begin{table}[!ht]
\centering
\begin{tabular}{c|p{14pt}|p{14pt}|p{14pt}|p{14pt}|p{14pt}|p{14pt}|p{14pt}}
\toprule
	$f$,Гц &&&&&&&\\
	\midrule
	$U_\textcyrillic{вх} $&&&&&&&\\
	\midrule
	$U_\textcyrillic{вых} $&&&&&&&\\
	\midrule
	$K=\textcyrillic{вых}/U_\textcyrillic{вх} $&&&&&&&\\
\bottomrule

\end{tabular}
	\label{a4x}
	\caption{Измерение параметров амплитудно-частотной характеристики фильтра}
\end{table}

\item Собрать схемы по рис. \ref{active_filter2}, \ref{active_filter3}, \ref{active_filter4}.

\item Снять и построить амплитудно-частотную характеристику фильтра при заданных параметрах элемента фильтра.
        Результаты измерений занести в таблицу \ref{a4x}

\item Изменить по указанию преподавателя параметры элементов фильтра. Снять (таблица \ref{a4x} ) и построить
	амплитудно-частотную характеристику фильтра при новых параметрах фильтра.

\item Построить графики АХЧ для приведенных схем с помощью частотного анализа.
\end{itemize}
\end{document}
