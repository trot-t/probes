\providecommand{\pdfxopts}{a-1b,cyrxmp}
\providecommand{\thisyear}{2020}
\immediate\write18{rm \jobname.xmpdata}%  uncomment for Unix-based systems
\begin{filecontents*}{\jobname.xmpdata}
\Title{Датчики. Лабораторная работа №5\textemdash\thisyear}
\Author{Артем Николаевич Прокшин}
\Creator{pdfTeX + pdfx.sty with options \pdfxopts }
\Subject{Лабораторная работа №5
\sep
        синхронизация с сетью}
\Keywords{операционный усилитель, синхронизация с сетью, ЛЭТИ}
\CoverDisplayDate{март \thisyear}
\CoverDate{2020-04-08}
\Copyrighted{True}
\Copyright{Public Domain}
\CopyrightURL{http://github.com/trot-t}
\Creator{pdfTeX + pdfx.sty with options \pdfxopts }
\end{filecontents*}

\documentclass[a4paper,12pt]{article}

\pdfcompresslevel=9

\usepackage[\pdfxopts]{pdfx}[2016/03/09]
\PassOptionsToPackage{obeyspaces}{url}
\let\tldocrussian=1  % for live4ht.cfg


\usepackage{extsizes}
\usepackage[left=15mm, top=20mm, right=15mm, bottom=20mm, nohead, footskip=7mm]{geometry} % настройки полей документа

\usepackage[T2A]{fontenc}
\usepackage[utf8]{inputenc}
\usepackage[english,russian]{babel}
\usepackage{tikz}
\usetikzlibrary{positioning}
\usepackage[european,cuteinductors,smartlabels,siunitx]{circuitikz}

%%% Межстрочный интервал
\usepackage{setspace}

%% таблицы
\usepackage{booktabs}
%multi-row
\usepackage{multirow}

%% для кода
\usepackage{color}
\usepackage{listingsutf8}

\definecolor{lightgrey}{rgb}{0.9,0.9,0.9}
\definecolor{lightblue}{rgb}{0,0,1}

\definecolor{grey}{rgb}{0.5,0.5,0.5}
\definecolor{blue}{rgb}{0,0,1}
\definecolor{violet}{rgb}{0.5,0,0.5}

\definecolor{darkred}{rgb}{0.5,0,0}
\definecolor{darkblue}{rgb}{0,0,0.5}
\definecolor{darkgreen}{rgb}{0,0.5,0}


\lstset{%
  language=C++,%
  morekeywords={constexpr,nullptr,size_t,uint32_t,assert,override,final},%
  basicstyle=\ttfamily\footnotesize,%
  sensitive=true,%
  keywordstyle=\color{blue},%
  stringstyle=\color{darkgreen},%
  commentstyle=\color{violet},%
  showstringspaces=false,%
  tabsize=2,%
  frame=leftline,
  rulecolor=\color{lightblue},
  xleftmargin=20pt,
}

\lstset{
%extendedchars=\true,
%inputencoding=utf8x,   
  numberstyle=\tiny,
  numbers=left,
  numbersep=10pt,
  xleftmargin=20pt,
  %framesep=4.5mm,
  %framexleftmargin=2.5mm,
  framexleftmargin=5pt,
  framesep=15pt,
  fillcolor=\color{lightgrey},
}


\title{Прецизионные схемы и малошумящая аппаратура}
\author{}
% Конец преамбулы
\begin{document}
%\maketitle
%\begin{tikzpicture}
%\newcommand{\xb}{-3}
%\newcommand{\xa}{3}
%\draw[thin, ->] (-6,0) -- (6,0) node[right] {$X$};
%\draw[thin, ->] (0,-6) -- (0,6) node[left] {$Y$};
%\foreach \x\xtext in {-5/-5,5/5,{\xb}/\xb,{\xa}/{\displaystyle \frac{-b+\sqrt{b^2-4ac}}{2a}}} % 
%   \draw (\x,0.1) -- (\x,-0.1) node[below] {$\xtext$};

 %\draw[domain=-5:5, help lines, smooth]
 %       plot ({\x},{0.2*(\x-\xa)*(\x-\xb)});
%\end{tikzpicture}
%\section{Прецизионные схемы и малошумящая аппаратура}
\section{Синхронизация с сетью}

{\it Цель:} получить сигнал, облегчающий синхронизацию с сетью.
 
\subsection{Задание на работу №5}

Для корректного отсчета длительности периодов сетевого напряжения на микроконтроллер подается сигнал синхронизации,
имеющий значение логической единицы по время отрицательного полупериода напряжения фазы А сети и логического
нуля в остальное время (см. \ref{Synchgraph})

\begin{figure}[ht!]
\centering
	\newcommand{\PI}{3.14159265}
\begin{circuitikz}
\draw[thin,->,>=latex] (0,0) -- (8,0) node[below] {$\omega t$};
	\draw[thin,->,>=latex] (0,0) -- (0,2.5) node[left] {$U_{A},U_{synch}$};
	\draw[domain=0:7,help lines, smooth, samples=400] 
	plot ({\x},{sin(3*\x r)})
;
	\draw[red,thick] (0,0) -- ({\PI/3},0) -- ({\PI/3},1.5) -- ({2*\PI/3},1.5) -- ({2*\PI/3},0) -- ({\PI},0) --  ({\PI},1.5) 
	-- ({4*\PI/3},1.5) -- ({4*\PI/3},0) -- ({5*\PI/3},0) -- ({5*\PI/3},1.5) -- ({2*\PI},1.5) -- ({2*\PI},0) -- ({7*\PI/3},0)
;
\end{circuitikz}
	\caption{Графики сигнала синхронизации(красный) и сигнала напряжения фазы}
	\label{Synchgraph}
\end{figure}


\begin{figure}[!ht]
\centering
\begin{circuitikz}[scale=0.9]
	\ctikzset{multipoles/dipchip/width=2.6}
\draw
	(0,0) to[R,a=\tiny{R1},l=\tiny{\SI{30}{k}}] ++(-3,0) to[short] node[midway, above] {\small{synch in}}  ++ (-0.3,0)
	(0,0) to[C, l=\tiny{C1},a=\tiny{\SI{0.1}{uF}},label/align=rotate] ++ (0,3) to[short] ++(3,0)
	(0,0) to[C,l=\tiny{C2},a=\tiny{\SI{0.1}{uF}},*-*] ++ (3,0) to[R,l=\tiny{R3},a=\tiny{\SI{60.4}{k}},label/align=rotate, -*] ++(0,3)
	(4.33,-1.5) node[op amp] (opamp1) {}  
	(opamp1.up) --++(0,0.5) node[vcc] {\small{5\,\textnormal{V}}}
	(opamp1.down) --++(0,-0.5) node[vee]{\small{-5\,\textnormal{V}}} 
	(0,-3) node(A) {} to[R,l=\tiny{R2},a=\tiny{\SI{30}{k}},label/align=rotate, *-] ++ (0,3) 
	(0,-3) node[ground] {}
	(A.center) -| (opamp1.+)
	(3,0) |- (opamp1.-)
	(3,3) -| (opamp1.out) to[R,l=\tiny{R4},a=\tiny{\SI{30}{k}},*-*] ++ (2,0) node (B) {}
	(opamp1.up) node[above right] {{\tiny DA1}}
	(opamp1.down) node[below right] {{\tiny OP275GS}}


	(10.6,-1.85) node[dipchip, hide numbers, no topmark, external pins width=0.0] (D) {}
	(D.bpin 1) node[right, font=\tiny]  {BAL/STRB} node[above left,font=\tiny] {6}
	(D.bpin 2) node[right, font=\tiny]  {IN+} node[above left, font=\tiny] {2}
	(D.bpin 3) node[right, font=\tiny]  {IN-} node[above left, font=\tiny] {3}
	(D.bpin 4) node[right, font=\tiny]  {BALANCE} node[above left, font=\tiny] {5}
        (D.bpin 5) node[above left, font=\tiny]  {EMIOUT}
        (D.bpin 8) node[below left, font=\tiny]  {COLOUT}
	(D.north) node[above] {{\tiny DA2}}
	(D.south) node[below] {{\tiny LM211}}

	(B.center) -| (D.bpin 2)
	($(D.bpin 8) ! 0.5 ! (D.bpin 7)$) node (COL) {} to[short] ++ (0.5,0) node (COLOUT) {}
	(COL) node[above right, font=\tiny] {7}

	($(D.bpin 6) ! 0.5 ! (D.bpin 5)$) node (EMI) {} to[short] ++ (0.5,0) node(EMIOUT){}
	(EMI) node[above right, font=\tiny] {1}
	(D.bpin 1) --++ (-0.6,0) |- (D.bpin 4)
	(D.bpin 3) --++ (-1,0) --++ (0,-1) node[ground] {}
	(B.center) --++ (0,2.5) node(C){} to[R,l=\tiny{R5},a=\tiny{\SI{1}{MEG}},*-*] (COLOUT |- C) node(D){} -- (COLOUT.center)
	(C.center) --++ (0,1.8) node(E){} to[C,l=\tiny{C1} \tiny{\SI{100}{pF}}] (COLOUT |- E) -- (D.center)
	(COLOUT) to[R,l=\tiny{R6},a=\tiny{\SI{4}{k}},*-] ++ (2.5,0) node[above] {{\small +5V}}
	(EMIOUT) ++(0,-2.2) node[ground] {} ++(0,-0.6) to[R,l=\tiny{R7},a=\tiny{\SI{30}{k}},label/align=rotate, ] (EMIOUT) 
	(EMIOUT) to[short,*-] ++(2.5,0) node[below left] {\small{synch out}}
	;
\begin{scope}
%	\ctikzset{tripoles/op amp/height=2.0, tripoles/op amp/width=1.4,}
	\ctikzset{tripoles/npn/width=0.5, tripoles/npn/base height=0.9, tripoles/npn/base width=0.9}
	\draw (10.3,-1.85) node[op amp, scale=0.5] (opamp2) {};
	\draw (11.4,-1.85) node[npn, scale=0.5] (npn1) {};
\end{scope}
	

\end{circuitikz}
	\caption{Принципиальная электрическая схема канала синхронизации с сетью.}
	\label{Synch}
\end{figure}

Сигнал с входного канала напряжения сети фазы А (см. \ref{Synch}) поступает на полосовой фильтр, настроенный на 50 Гц, 
выполненный на операционном усилителе DA1, резисторах R1, R2, R3, конденсаторах C1 и C2 для снижения вероятности ложной синхронизации 
при сильно искаженном сетевом напряжении. Частота полосы пропускания фильтра рассчитывается следующим образом:

\begin{equation}
	f = \frac{1}{2\pi C} \sqrt{\frac{R_1+R_2}{R_1\cdot R_2 \cdot R_3}}
\label{f}
\end{equation}

 Далее сигнал поступает на компаратор DA2, имеющий \textcolor{red}{отрицательную} обратную связь для обеспечения гистерезиса и исключения дребезга 
 при переходе сигнала через "$0$"\,, где сравнивается с уровнем "0"\  и при напряжении сигнала ниже "0"\  компаратор формирует на выходе логическую "1". 
 Поскольку выход компаратора выполнен по схеме с "открытым коллектором"\,, на коллектор подается +5 В, 
 а выходной сигнал поступает с эмиттера на логический вход микроконтроллера.

\subsection{Индивидуальные задания}

Собрать схему в tina. Вход {\it synch in} соответствует выходу в схеме к лабораторной работе №4. 

\begin{itemize}
	\item Построить графики переходных процессов.

	\item Привести вывод формулы (\ref{Synch}) для полосового фильтра.

	\item Верно ли употреблено выделенное красным слово в тексте или следует заменить его словом \textcolor{red}{положительную}?

	\item Проверить влияние на сигнал синхронизации 3-й гармоники с амплитудой равной амплитуде основной гармонике, 
		с амплитудой равной $\frac{1}{2}, \frac{1}{3}, \frac{1}{10}$ долям от амплитуды основной гармоники.

		К основной гармонике подмешивать $K$-е гармоники по правилу:
		$$
		A\cdot\sin(\omega t) + B\cdot\sin(K\cdot\omega t + \varphi_{\textnumero\textcyrillic{варианта}})
		$$
		где $K$ -- номер гармоники, A, B -- амплитуды основной и подмешиваемой гармоники, $\varphi_{\textnumero\textcyrillic{варианта}}$ -- начальная фаза в зависимости от $N$ -- номера по списку в группе .

		${\displaystyle \varphi_{\textnumero\textcyrillic{варианта}} = \frac{2\pi}{30\cdot K} \left(N-1\right)}$ 

	\item Аналогично проверить влияние на сигнал синхронизации 5-й, 7-й, 11-й, 13-й гармоник.

	\item Построить АЧХ схемы.
\end{itemize}
%\section*{Литература}
\renewcommand{\bibname}{}
\begin{thebibliography}{3} 
	\bibitem{lm211} \href{https://www.ti.com/product/LM211}{LM211 Differential Comparator With Strobes, PSpice Model}
\end{thebibliography}

\end{document}
