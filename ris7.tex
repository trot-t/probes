\documentclass[utf8x, times, 14pt]{extarticle}
%\documentclass[utf8x, times, 14pt]{extstandalone}
%\documentclass[utf8x, times, 14pt]{standalone}
\usepackage[fontsize=12]{fontsize}
%\usepackage[active,tightpage]{preview}

\usepackage[T2A]{fontenc}
\usepackage[utf8]{inputenc}
\usepackage[english,russian]{babel}
\usepackage{tikz}
\usetikzlibrary{positioning}
\usepackage[european,cuteinductors,smartlabels,siunitx]{circuitikz}

\usepackage[a4paper, margin=0.0cm]{geometry}

\begin{document}

\pdfpagewidth 3.0in
\pdfpageheight 2.4in


\hspace{-6mm}
\begin{tikzpicture}
        \draw
        (0,0) node[op amp, yscale=-1] (opamp) {}
        (opamp.-) --++ (-1,0) --++ (0,-1.2) node(A) {} to[R,l_={$R_1$}] ++ (0,-3) node[tground] (G) {} % инв. вход ОУ соединим с землей резистором R1
        (opamp.out) to[short,*-] (opamp.out |- A) --++ (-0.7, 0) node(B) {}
        (A.center) to[short,*-] ++ (0.7,0) node(C){} to[R,l_=$R_{\textcyrillic{ОС}}$] (B.center)
        (C.center) to[short,*-] ++ (0,-1.3) node(D) {}  to[C,l_=$C$] (B |- D) to[short,-*] (B.center)
        (opamp.out) to[short,-o] ++ (1,0) node(OUT_P) {}       % вывод ОУ вместе с кружком "-o"
        (OUT_P |- G) node[tground] {} to[short,-o] ++ (0,3) node(OUT_M) {} % от земли до выхода схемы
        ($(OUT_P) !0.5! (OUT_M)$) node {$U_\textcyrillic{вых}$}  % справа от точки посередине между выходами схемы
        (opamp.+) to[short,-o] ++ (-3,0) node(IN_P) {}  % инв.вход ОУ соединяем со входом "-o" схемы резистором R1
        (IN_P |- G) node[tground] {} to[short,-o] ++ (0,3.5) node(IN_M) {}
        ($(IN_P) !0.5! (IN_M)$) node {$U_\textcyrillic{вх}$} % посередине между узлами (+) и (-) входов -- подпись к метке
        ;  % всегда в конце команды(в данном случае \draw) ставим точку с запятой
        % !!! это самая распространенная ошибка почему может не работать
\end{tikzpicture}

\end{document}

